%%%%%%%%%%%%%%%%%
% This is an sample CV template created using altacv.cls
% (v1.1.2, 1 February 2017) written by LianTze Lim (liantze@gmail.com). Now compiles with pdfLaTeX, XeLaTeX and LuaLaTeX.
%
%% It may be distributed and/or modified under the
%% conditions of the LaTeX Project Public License, either version 1.3
%% of this license or (at your option) any later version.
%% The latest version of this license is in
%%    http://www.latex-project.org/lppl.txt
%% and version 1.3 or later is part of all distributions of LaTeX
%% version 2003/12/01 or later.
%%%%%%%%%%%%%%%%

%% If you need to pass whatever options to xcolor
\PassOptionsToPackage{dvipsnames}{xcolor}

%% If you are using \orcid or academicons
%% icons, make sure you have the academicons
%% option here, and compile with XeLaTeX
%% or LuaLaTeX.
% \documentclass[10pt,a4paper,academicons]{altacv}

%% Use the "normalphoto" option if you want a normal photo instead of cropped to a circle
% \documentclass[10pt,a4paper,normalphoto]{altacv}

\documentclass[10pt,a4paper]{altacv}

%% AltaCV uses the fontawesome and academicon fonts
%% and packages.
%% See texdoc.net/pkg/fontawecome and http://texdoc.net/pkg/academicons for full list of symbols.
%%
%% Compile with LuaLaTeX for best results. If you
%% want to use XeLaTeX, you may need to install
%% Academicons.ttf in your operating system's font
%% folder.


% Change the page layout if you need to
\geometry{left=1cm,right=9cm,marginparwidth=6.8cm,marginparsep=1.2cm,top=1.25cm,bottom=1.25cm}

% Change the font if you want to.

% If using pdflatex:
\usepackage[utf8]{inputenc}
\usepackage[T1]{fontenc}
\usepackage[default]{lato}

% If using xelatex or lualatex:
% \setmainfont{Lato}

% Change the colours if you want to
\definecolor{Mulberry}{HTML}{72243D}
\definecolor{SlateGrey}{HTML}{2E2E2E}
\definecolor{LightGrey}{HTML}{666666}
\definecolor{VividPurple}{HTML}{3E0097}
\definecolor{SlateGrey}{HTML}{2E2E2E}
\definecolor{LightGrey}{HTML}{666666}
\definecolor{DarkBlue}{HTML}{1c304b}
\definecolor{LightBlue}{HTML}{2079c7}

\colorlet{heading}{DarkBlue}
\colorlet{heading-line}{LightBlue}
\colorlet{accent}{DarkBlue}
\colorlet{emphasis}{SlateGrey}
\colorlet{body}{LightGrey}

% Change the bullets for itemize and rating marker
% for \cvskill if you want to
\renewcommand{\itemmarker}{{\small\textbullet}}
\renewcommand{\ratingmarker}{\faCircle}

%% sample.bib contains your publications
\addbibresource{sample.bib}

\begin{document}
\name{AYUSH GOEL}
\tagline{Software Engineer}
\photo{2.8cm}{avatar}
\personalinfo{%
  % Not all of these are required!
  % You can add your own with \printinfo{symbol}{detail}
  \email{ag.goelayush@gmail.com}
  \phone{(315) 572-6532}
  % \mailaddress{Address, Street, 00000 County}
  % \location{Location, COUNTRY}
  \homepage{ayushgoel.me}
  % \twitter{@twitterhandle}
  % \linkedin{linkedin.com/in/ayush-goel-589ba215a/}
  \github{ag-ayush}
  %% You MUST add the academicons option to \documentclass, then compile with LuaLaTeX or XeLaTeX, if you want to use \orcid or other academicons commands.
%   \orcid{orcid.org/0000-0000-0000-0000}
}

%% Make the header extend all the way to the right, if you want.
\begin{fullwidth}
\makecvheader
\end{fullwidth}

%% Provide the file name containing the sidebar contents as an optional parameter to \cvsection.
%% You can always just use \marginpar{...} if you do
%% not need to align the top of the contents to any
%% \cvsection title in the "main" bar.
\cvsection[resume-sidebar]{Work Experience}

\cveventnew{Student Affairs at R.I.T.}{Orientation Leader}{Responsible for facilitating the transition of a group of first year and transfers students representing multiple programs from across the university during New Student Orientation.}{August 2018}{Rochester, NY}

% \divider

\cvsection{Education}
\cveventnew{Rochester Institute of Technology}{Bachelor of Science: \newline Double Major in Computer Science and Computational Mathematics}{\textbf{Cumalitive GPA:} 3.8/4.0 \newline \textbf{Awards:} Dean's List (2)}{2017 - 2022}{}

\cvsection{Projects}

\cvevent{Smart Shades}{github.com/ag-ayush/smart-window-shades}{}{}
A web service, written using Python flask, bootstrap, JavaScript and ajax, that allows one to control their window shades.
The physical shades system was built by using a Raspberry Pi and a stepper motor.
\divider

\cvevent{Project 2}{Funding agency/institution}{Project duration}{}
A short abstract would also work.

\medskip

\end{document}
