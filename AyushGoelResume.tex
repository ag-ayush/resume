%%%%%%%%%%%%%%%%%%%%%%%%%%%%%%%%%%%%%%%
% Deedy - One Page Two Column Resume
% LaTeX Template
% Version 1.2 (16/9/2014)
%
% Original author:
% Debarghya Das (http://debarghyadas.com)
%
% Original repository:
% https://github.com/deedydas/Deedy-Resume
%
% IMPORTANT: THIS TEMPLATE NEEDS TO BE COMPILED WITH XeLaTeX
%
% This template uses several fonts not included with Windows/Linux by
% default. If you get compilation errors saying a font is missing, find the line
% on which the font is used and either change it to a font included with your
% operating system or comment the line out to use the default font.
%
%%%%%%%%%%%%%%%%%%%%%%%%%%%%%%%%%%%%%%

% !TEX program = xelatex

\documentclass[]{deedy-resume-openfont}
\usepackage{fancyhdr}

\pagestyle{fancy}
\fancyhf{}
\renewcommand{\headrulewidth}{0pt}

\begin{document}

%%%%%%%%%%%%%%%%%%%%%%%%%%%%%%%%%%%%%%
%
%     LAST UPDATED DATE
%
%%%%%%%%%%%%%%%%%%%%%%%%%%%%%%%%%%%%%%
\lastupdated

%%%%%%%%%%%%%%%%%%%%%%%%%%%%%%%%%%%%%%
%
%     TITLE NAME
%
%%%%%%%%%%%%%%%%%%%%%%%%%%%%%%%%%%%%%%
\namesection{Ayush}{Goel}{ \href{http://ayushgoel.com}{ayushgoel.com} | \href{mailto:ag.goelayush@gmail.com}{ag.goelayush@gmail.com} | 315.572.6532
\\ \smallskip
Github:// \href{https://github.com/ag-ayush}{\bf ag-ayush} |
LinkedIn://  \href{https://www.linkedin.com/in/ag-ayush}{\bf ag-ayush}
}

%%%%%%%%%%%%%%%%%%%%%%%%%%%%%%%%%%%%%%
%
%     COLUMN ONE
%
%%%%%%%%%%%%%%%%%%%%%%%%%%%%%%%%%%%%%%

\begin{minipage}[t]{0.30\textwidth}

%%%%%%%%%%%%%%%%%%%%%%%%%%%%%%%%%%%%%%
%     OBJECTIVE
%%%%%%%%%%%%%%%%%%%%%%%%%%%%%%%%%%%%%%

\section{Objective}
\textbf{Seeking a full-time \\
opportunity starting August 2022.}

%%%%%%%%%%%%%%%%%%%%%%%%%%%%%%%%%%%%%%
%     EDUCATION
%%%%%%%%%%%%%%%%%%%%%%%%%%%%%%%%%%%%%%

\section{Education}

% \runsubsection{Rochester Institute of Technology } \\

% \vspace{\topsep} % Hacky fix for awkward extra vertical space

\runsubsection{Rochester Institute of Technology } \\
\descript{MS in Computer Science}
\location{2021-2022 \\
GPA: 4.00}
\descript{BS in Computer Science \\
BS in Computational Mathematics}
\location{2017-2021 \\
GPA: 3.79}
% \sectionsep

%%%%%%%%%%%%%%%%%%%%%%%%%%%%%%%%%%%%%%
%     LINKS
%%%%%%%%%%%%%%%%%%%%%%%%%%%%%%%%%%%%%%

% \section{Links}
% Github:// \href{https://github.com/ayush}{\bf ag-ayush} \\
% LinkedIn://  \href{https://www.linkedin.com/in/ag-ayush}{\bf ag-ayush} \\

%%%%%%%%%%%%%%%%%%%%%%%%%%%%%%%%%%%%%%
%     COURSEWORK
%%%%%%%%%%%%%%%%%%%%%%%%%%%%%%%%%%%%%%

% \section{Coursework}
% \textbf{Web Services and Service Oriented Computing \\
%   NoSQL and NewSQL Databases \\
%   Foundations of Computer Networks \\
%   Computational Complexity
% }

%%%%%%%%%%%%%%%%%%%%%%%%%%%%%%%%%%%%%%
%     SKILLS
%%%%%%%%%%%%%%%%%%%%%%%%%%%%%%%%%%%%%%

\section{Skills}
\subsection{Languages}
\textbf{Java \textbullet{} Python \textbullet{} Bash \textbullet{} MySQL
Groovy\\}

\vspace{\topsep} % Hacky fix for awkward extra vertical space

\subsection{Tools}
\textbf{git \textbullet{} Linux \textbullet{} Spring Boot \textbullet{} Maven
Docker \textbullet{} Kubernetes  \textbullet{} Jenkins
Splunk \textbullet{} DynamoDb \textbullet{} Neo4j 
Confluence \textbullet{} Tableau}
% Bootstrap, 
\sectionsep

%%%%%%%%%%%%%%%%%%%%%%%%%%%%%%%%%%%%%%
%     ACTIVITIES
%%%%%%%%%%%%%%%%%%%%%%%%%%%%%%%%%%%%%%

\section{Activities}

\subsection{Theta Chi Fraternity}
\descript{February 2021 - Present}
\location{A fraternity who extends a helping hand to the community by volunteering at local organizations and emphasizes focus on academics.}

\vspace{\topsep} % Hacky fix for awkward extra vertical space

\subsection{Computer Science House}
\descript{September 2017 - Present}
\descript{Evaluations Director 2019}
\location{A university community of computing centered people focused on technical and social development of its members.}

% \vspace{\topsep} % Hacky fix for awkward extra vertical space

% \subsection{FRC\textregistered Team 174}
% \descript{September 2015 - May 2017}
% \location{Led the team as the project manager, programming lead, and drive team coach.}

% \subsection{CS Ambassador}
% \descript{September 2018 - Present}
% \location{Represent the department of computer science at RIT; developed strong leadership and communication skills.}

\sectionsep


\section{Awards}
\begin{tabular}{ll}
2021         & CS Alumni Scholarship \\
2019         & NRS Scholar \\
2017-2020	   & Dean's List at RIT \\
% 2017	       & Best External Data at DataFest, RIT \\
\end{tabular}
\sectionsep

%%%%%%%%%%%%%%%%%%%%%%%%%%%%%%%%%%%%%%
%
%     COLUMN TWO
%
%%%%%%%%%%%%%%%%%%%%%%%%%%%%%%%%%%%%%%

\end{minipage}
\hfill
\begin{minipage}[t]{0.67\textwidth}

%%%%%%%%%%%%%%%%%%%%%%%%%%%%%%%%%%%%%%
%     EXPERIENCE
%%%%%%%%%%%%%%%%%%%%%%%%%%%%%%%%%%%%%%

\section{Experience}
\runsubsection{Intuit}
\descript{| Software Engineering Intern }
\location{May 2021 - August 2021 | Remote}
\vspace{\topsep} % Hacky fix for awkward extra vertical space
\begin{tightemize}
  \item Utilized \textbf{ArgoCD}, \textbf{Jenkins}, \textbf{Terraform}, and Intuit \textbf{Kubernetes} to create a CI/CD strategy and pattern based on Canary Deployment with \textbf{Service Mesh} for services on Identity 2.0 Platform.
  \item Organized a \textbf{successful} plan to increase developer velocity across multiple teams by analyzing the data metrics from their service deployments. Implemented \textbf{Agile improvements} such as Kanban and created several POCs utilizing new technology to increase developer confidence in deployments.
\end{tightemize}
\sectionsep

\runsubsection{Intuit}
\descript{| Software Engineering Co-Op }
\location{May 2019 - December 2019 | San Diego, CA}
\vspace{\topsep} % Hacky fix for awkward extra vertical space
\begin{tightemize}
  \item Back-end software engineer on the Identity Platform.
  \item Migrated 2 REST services from deploying through Spinnaker to utilizing \textbf{Docker} and \textbf{Kubernetes} with 0 downtime.
  \item Created tools to semi-automate the migration for other teams.
  \item Collaborated across teams to create the stack of a new back-end service.
  \item Utilized \textbf{Java} with \textbf{Spring Boot} and \textbf{Maven}, and used \textbf{Groovy} for scripts.
\end{tightemize}
\sectionsep

\runsubsection{Rochester Institute of Technology}
\descript{| Multiple Positions}
\location{ \textbf{Graduate Teaching Assistant} | August 2021 - Present}
\begin{tightemize}
  \item Mentor First-Year students with the core computer science courses and help debug their code for homework/lab/projects.
  \item Collaborate with professors, prepare and review course material, and hold 6 weekly review sessions.
\end{tightemize}
\location{ \textbf{Research Software Engineer} | May 2020 - May 2021}
\begin{tightemize}
  \item Collaboratively developed, tested, deployed, and documented ASSERT, a \textbf{Java} and \textbf{Python} application to continuously synthesize and separate cyberattack behavior models to enable better prediction of future actions.
  \item Transformed the research prototype to a leaner and better containerized application.
  \item Developed dynamic visualization dashboards using \textbf{Python}, \textbf{Dash} and \textbf{Plotly} for analyzing models created by ASSERT.
\end{tightemize}
\location{ \textbf{Supplemental Instruction Program Assistant} | January 2020 - December 2020}
\begin{tightemize}
  \item Collaboratively helped run the Supplemental Instruction (SI) program at RIT.
  \item Assisted with the planning and facilitating of new hires and weekly training.
  \item Observed and provided constructive feedback to SI leaders.
\end{tightemize}
\location{ \textbf{Orientation Coordinator} | January 2020 - October 2020}
\begin{tightemize}
  \item Collaborated across departments to organize and develop Orientation Program at RIT for over 3000 students.
  \item Managed, mentored and trained 20 Orientation Supervisors and 150 Orientation Leaders.
  \item Developed strong leadership, organization, communication, and team building skills.
\end{tightemize}

% \location{ \textbf{Supplemental Instruction Leader} | August 2018 - May 2019}
% \begin{tightemize}
%   \item Aid students in historically difficult CS (Python, Java) courses.
%   \item Plan and facilitate weekly review sessions for the students in order to further their understanding and confidence in the language, thereby also refining my own knowledge of the fundamentals.
% \end{tightemize}

% \location{ \textbf{Orientation Supervisor} | August 2019}
% \begin{tightemize}
%   \item Trained 20 Orientation Leaders to lead a group of new students and helped facilitate the transition of 400 students.
% \end{tightemize}

%%%%%%%%%%%%%%%%%%%%%%%%%%%%%%%%%%%%%%
%     Projects
%%%%%%%%%%%%%%%%%%%%%%%%%%%%%%%%%%%%%%

% \section{Projects}
% \runsubsection{Ride Board}
% \location{\href{https://www.github.com/ag-ayush/rideboard}{github.com/ag-ayush/rideboard}}
% A web application developed for hosting large-scale events during which users would like to carpool to reach their destination.
% The app was built using \textbf{Flask-SQLAlchemy}, \textbf{PostgreSQL}, \textbf{Python Flask}, \textbf{Flask-pyoidc} and \textbf{bootsrap}, and deployed using OpenShift.
% I learned about relational databases using SQLAlchemy, database design, and hosting databases with phpPgAdmin.
% \sectionsep

% \runsubsection{Ride Board API}
% \location{\href{https://github.com/ag-ayush/RideBoardAPI}{github.com/ag-ayush/RideBoardAPI}}
% A \textbf{RESTful} HTTP API developed for the CSH Ride Board application using \textbf{Python Flask}, \textbf{Flask-SQLAlchemy}, and \textbf{Flask-pyoidc}. The responses are produced in \textbf{JSON} format. I learned about APIs and parsing JSON.
% \sectionsep

% \runsubsection{Smart Shades}
% \location{\href{https://www.github.com/ag-ayush/smart-window-shades}{github.com/ag-ayush/smart-window-shades}}
% A web service, written with \textbf{Python Flask}, \textbf{bootstrap}, \textbf{JavaScript} and \textbf{AJAX}, that allows one to control their window shades. The physical shades system was built with a Raspberry Pi and a stepper motor. I learned about JavaScript, programming using RaspberryPi and working with hardware.
% \sectionsep

% \runsubsection{Big Data and Employment}
% \location{}
% Developed at ASA DataFest 2018, this was a \textbf{Java} application built to process datasets in CSV format.
% The application took the large dataset from Indeed and custom datasets found online as input to produce the states that would be the best for you to work in based on your job interest, expereince, and other factors.
% \sectionsep

% \runsubsection{WebCheckers}
% \location{}
% A \textbf{Java}-based web server for the game of checkers that allows users to play against eachother, against a computer, and spectate games. It utilizes the \textbf{Spark} web micro framework and the \textbf{FreeMarker} template engine.
% \sectionsep

% \runsubsection{Lane Follow Bot}
% \location{\href{https://www.github.com/ag-ayush/lane-follow-bot}{github.com/ag-ayush/lane-follow-bot}}
% A small software program written using \textbf{Python} and \textbf{OpenCV} to detect white lines on green grass and then output which direction to move in order to stay between the two lines. I learned how to work with basics of OpenCV.
% \sectionsep

%%%%%%%%%%%%%%%%%%%%%%%%%%%%%%%%%%%%%%
%     Volunteer
%%%%%%%%%%%%%%%%%%%%%%%%%%%%%%%%%%%%%%

% \section{Volunteer Work}
%
% \runsubsection{FIRST\textregistered \space Robotics Finger Lakes Regional}
% \location{Various Roles}
% % Helped set up and staff a 3 day FIRST\textregistered Robotics Competition regional.
% \sectionsep
%
% \runsubsection{Elementary School Tutoring}
% \location{Tutor}
% % Tutored elementary students in mathematics and science.
% \sectionsep


\sectionsep

\end{minipage}
\end{document}  \documentclass[]{article}
