%%%%%%%%%%%%%%%%%
% This is an sample CV template created using altacv.cls
% (v1.1.2, 1 February 2017) written by LianTze Lim (liantze@gmail.com). Now compiles with pdfLaTeX, XeLaTeX and LuaLaTeX.
%
%% It may be distributed and/or modified under the
%% conditions of the LaTeX Project Public License, either version 1.3
%% of this license or (at your option) any later version.
%% The latest version of this license is in
%%    http://www.latex-project.org/lppl.txt
%% and version 1.3 or later is part of all distributions of LaTeX
%% version 2003/12/01 or later.
%%%%%%%%%%%%%%%%%

%% If you need to pass whatever options to xcolor
\PassOptionsToPackage{dvipsnames}{xcolor}

%% If you are using \orcid or academicons
%% icons, make sure you have the academicons
%% option here, and compile with XeLaTeX
%% or LuaLaTeX.
% \documentclass[10pt,a4paper,academicons]{altacv}

%% Use the "normalphoto" option if you want a normal photo instead of cropped to a circle
\documentclass[10pt,a4paper,normalphoto]{altacv}

%% AltaCV uses the fontawesome and academicon fonts and packages.
%% See texdoc.net/pkg/fontawecome and http://texdoc.net/pkg/academicons for full list of symbols.
%%
%% Compile with LuaLaTeX for best results. If you
%% want to use XeLaTeX, you may need to install
%% Academicons.ttf in your operating system's font
%% folder.


% Change the page layout if you need to
\geometry{left=1cm,right=9cm,marginparwidth=6.8cm,marginparsep=1.2cm,top=1.25cm,bottom=1.25cm}

% Change the font if you want to.
  % If using pdflatex:
  \usepackage[utf8]{inputenc}
  \usepackage[T1]{fontenc}
  \usepackage[default]{lato}

  % If using xelatex or lualatex:
  % \setmainfont{Lato}

% Change the colours if you want to
\definecolor{Mulberry}{HTML}{72243D}
\definecolor{SlateGrey}{HTML}{2E2E2E}
\definecolor{VividPurple}{HTML}{3E0097}
\definecolor{SlateGrey}{HTML}{2E2E2E}
\definecolor{LightGrey}{HTML}{666666}
\definecolor{DarkBlue}{HTML}{17273d}
\definecolor{LightBlue}{HTML}{155185}

\colorlet{heading}{DarkBlue}
\colorlet{heading-line}{LightBlue}
\colorlet{accent}{DarkBlue}
\colorlet{emphasis}{SlateGrey}
\colorlet{body}{LightGrey}

% Change the bullets for itemize and rating marker
% for \cvskill if you want to
\renewcommand{\itemmarker}{{\small\textbullet}}
\renewcommand{\ratingmarker}{\faCircle}

%% sample.bib contains your publications
% \addbibresource{sample.bib}

\begin{document}
\name{AYUSH GOEL}
\tagline{Software Engineer}
% \photo{2.8cm}{profile-square}
\personalinfo{%
  % You can add your own with \printinfo{symbol}{detail}
  \email{\href{mailto:ag.goelayush@gmail.com}{ag.goelayush@gmail.com}}
  \phone{(315) 572-6532}
  \homepage{\href{https://ayushgoel.com}{ayushgoel.com}}
  % \twitter{@twitterhandle}
  \github{\href{https://www.github.com/ag-ayush}{ag-ayush}}
  \linkedin{\href{https://www.linkedin.com/in/ag-ayush/}{ag-ayush}}
  %% You MUST add the academicons option to \documentclass, then compile with LuaLaTeX or XeLaTeX, if you want to use \orcid or other academicons commands.
%   \orcid{orcid.org/0000-0000-0000-0000}
}

%% Make the header extend all the way to the right, if you want.
\begin{fullwidth}
  \makecvheader
\end{fullwidth}

%% Provide the file name containing the sidebar contents as an optional parameter to \cvsection.
%% You can always just use \marginpar{...} if you do
%% not need to align the top of the contents to any
%% \cvsection title in the "main" bar.
\cvsection[resume-sidebar]{Work Experience}
  \cveventnew{Academic Support Center}{Supplemental Instruction Leader}{Responsible for planning, scheduling, and facilitating weekly study sessions for CSCI-141 students.}{August 2018 - Present}{Rochester, NY}

  \divider

  \cveventnew{Student Affairs at R.I.T}{Orientation Leader}{Responsible for facilitating the transition of a group of first year and transfers students representing multiple programs from across the university during New Student Orientation.}{August 2018}{Rochester, NY}
  % \divider


\cvsection{Projects}
  \cvevent{Ride Board API}{\href{https://github.com/ag-ayush/RideBoardAPI}{github.com/ag-ayush/RideBoardAPI}}{}{}
  A \textbf{RESTful} HTTP API developed for the CSH Ride Board application using \textbf{Python Flask}, \textbf{Flask-SQLAlchemy}, and \textbf{Flask-pyoidc}. The responses are produced in \textbf{JSON} format.

  \divider

  \cvevent{Ride Board}{\href{https://www.github.com/ag-ayush/rideboard}{github.com/ag-ayush/rideboard}}{}{}
  A web application developed for hosting large-scale events during which users would like to carpool in order to reach their destination.
  The app was built using the RideBoard1API, \textbf{Python Flask}, \textbf{Flask-pyoidc} and \textbf{bootsrap}.

  \divider

  \cvevent{Smart Shades}{\href{https://www.github.com/ag-ayush/smart-window-shades}{github.com/ag-ayush/smart-window-shades}}{}{}
  A web service, written with \textbf{Python Flask}, \textbf{bootstrap}, \textbf{JavaScript} and \textbf{ajax}, that allows one to control their window shades.
  The physical shades system was built with a Raspberry Pi and a stepper motor.

  \divider

  \cvevent{Where should you work?}{\textit{\small{Link not available due to competition regulations.}}}{}{}
  Developed at ASA DataFest 2018, this was a \textbf{Java} application built to process datasets in CSV format.
  The application took the large dataset from Indeed and custom datasets found online as input to produce the states that would be the best for you to work in based on your job interest, expereince, and other factors.
  Won best use of external data.

  \divider

  \cvevent{Lane Follow Bot}{\href{https://www.github.com/ag-ayush/lane-follow-bot}{github.com/ag-ayush/lane-follow-bot}}{}{}
  A small software program written using \textbf{Python} and \textbf{OpenCV} to detect white lines on green grass and then output which direction to move in order to stay between the two lines.
\medskip

\end{document}
