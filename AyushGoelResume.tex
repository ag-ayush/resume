%%%%%%%%%%%%%%%%%%%%%%%%%%%%%%%%%%%%%%%
% Deedy - One Page Two Column Resume
% LaTeX Template
% Version 1.2 (16/9/2014)
%
% Original author:
% Debarghya Das (http://debarghyadas.com)
%
% Original repository:
% https://github.com/deedydas/Deedy-Resume
%
% IMPORTANT: THIS TEMPLATE NEEDS TO BE COMPILED WITH XeLaTeX
%
% This template uses several fonts not included with Windows/Linux by
% default. If you get compilation errors saying a font is missing, find the line
% on which the font is used and either change it to a font included with your
% operating system or comment the line out to use the default font.
%
%%%%%%%%%%%%%%%%%%%%%%%%%%%%%%%%%%%%%%


\documentclass[]{deedy-resume-openfont}
\usepackage{fancyhdr}

\pagestyle{fancy}
\fancyhf{}
\renewcommand{\headrulewidth}{0pt}

\begin{document}

%%%%%%%%%%%%%%%%%%%%%%%%%%%%%%%%%%%%%%
%
%     LAST UPDATED DATE
%
%%%%%%%%%%%%%%%%%%%%%%%%%%%%%%%%%%%%%%
\lastupdated

%%%%%%%%%%%%%%%%%%%%%%%%%%%%%%%%%%%%%%
%
%     TITLE NAME
%
%%%%%%%%%%%%%%%%%%%%%%%%%%%%%%%%%%%%%%
\namesection{Ayush}{Goel}{ \href{http://ayushgoel.com}{ayushgoel.com} | \href{mailto:ag.goelayush@gmail.com}{ag.goelayush@gmail.com} | 315.572.6532
}

%%%%%%%%%%%%%%%%%%%%%%%%%%%%%%%%%%%%%%
%
%     COLUMN ONE
%
%%%%%%%%%%%%%%%%%%%%%%%%%%%%%%%%%%%%%%

\begin{minipage}[t]{0.30\textwidth}

%%%%%%%%%%%%%%%%%%%%%%%%%%%%%%%%%%%%%%
%     OBJECTIVE
%%%%%%%%%%%%%%%%%%%%%%%%%%%%%%%%%%%%%%

\section{Objective}
\textbf{Seeking a co-op / internship \\
opportunity in Summer 2020 / Spring 2021.}

%%%%%%%%%%%%%%%%%%%%%%%%%%%%%%%%%%%%%%
%     EDUCATION
%%%%%%%%%%%%%%%%%%%%%%%%%%%%%%%%%%%%%%

\section{Education}

\runsubsection{Rochester Institute of Technology } \\
\descript{B.S. in Computer Science \\
B.S. in Computational Mathematics}
\location{2017-2021 \\
GPA: 3.72}
% \sectionsep

%%%%%%%%%%%%%%%%%%%%%%%%%%%%%%%%%%%%%%
%     LINKS
%%%%%%%%%%%%%%%%%%%%%%%%%%%%%%%%%%%%%%

\section{Links}
Github:// \href{https://github.com/ayush}{\bf ag-ayush} \\
LinkedIn://  \href{https://www.linkedin.com/in/ag-ayush}{\bf ag-ayush} \\

%%%%%%%%%%%%%%%%%%%%%%%%%%%%%%%%%%%%%%
%     COURSEWORK
%%%%%%%%%%%%%%%%%%%%%%%%%%%%%%%%%%%%%%

\section{Coursework}
\textbf{Analysis of Algorithms \\
Concepts of Computer Systems \\
Mechanics of Programming \\
Principles of Data Management \\
}
% {\footnotesize \textit{\textbf{(Research Asst. \& Teaching Asst 2x) }}} \\
% Unix Tools and Scripting \\

%%%%%%%%%%%%%%%%%%%%%%%%%%%%%%%%%%%%%%
%     SKILLS
%%%%%%%%%%%%%%%%%%%%%%%%%%%%%%%%%%%%%%

\section{Skills}
\subsection{Languages}
\textbf{Java \textbullet{} Python \textbullet{} Bash \textbullet{} MySQL
Groovy \textbullet{} C \\}

\vspace{\topsep} % Hacky fix for awkward extra vertical space

\subsection{Tools}
\textbf{ git \textbullet{} Linux \textbullet{} Spring Boot \textbullet{} Maven
Docker \textbullet{} Kubernetes  \textbullet{} Jenkins
Splunk \textbullet{} DynamoDb \textbullet{} Flask}
% Bootstrap, 
\sectionsep

%%%%%%%%%%%%%%%%%%%%%%%%%%%%%%%%%%%%%%
%     ACTIVITIES
%%%%%%%%%%%%%%%%%%%%%%%%%%%%%%%%%%%%%%

\section{Activities}

\subsection{Computer Science House}
\descript{September 2018 - Present}
\descript{Evaluations Director 2019}
\location{A university community of computing centred people focused on technical and social development of its members.}

% \vspace{\topsep} % Hacky fix for awkward extra vertical space

% \subsection{FRC\textregistered Team 174}
% \descript{September 2015 - May 2017}
% \location{Led the team as the project manager, programming lead, and drive team coach.}

\vspace{\topsep} % Hacky fix for awkward extra vertical space

\subsection{CS Ambassador}
\descript{September 2015 - May 2017}
\location{One of the students representing the Computer Science department at RIT; developed strong leadership and communication skills.}

\sectionsep

%%%%%%%%%%%%%%%%%%%%%%%%%%%%%%%%%%%%%%
%
%     COLUMN TWO
%
%%%%%%%%%%%%%%%%%%%%%%%%%%%%%%%%%%%%%%

\end{minipage}
\hfill
\begin{minipage}[t]{0.67\textwidth}

%%%%%%%%%%%%%%%%%%%%%%%%%%%%%%%%%%%%%%
%     EXPERIENCE
%%%%%%%%%%%%%%%%%%%%%%%%%%%%%%%%%%%%%%

\section{Experience}
\runsubsection{Intuit}
\descript{| Software Engineering Co-Op }
\location{May 2019 - December 2019 | San Diego, CA}
\vspace{\topsep} % Hacky fix for awkward extra vertical space
\begin{tightemize}\item Back end software engineer for the IdProofing Team
  \item Migrated the service from deploying via Spinnaker to utilizing \textbf{Docker} and \textbf{Kubernetes}
  \item Created a REST service to create, send and confirm OTP
  \item Worked primarily in \textbf{Java} using \textbf{Spring Boot} and \textbf{Maven}
\end{tightemize}
\sectionsep

\runsubsection{Academic Support Center}
\descript{| Supplemental Instruction Leader }
\location{August 2018 - Present | Rochester, NY}
% \vspace{\topsep} % Hacky fix for awkward extra vertical space
\begin{tightemize}\item Aid students in historically difficult CS (Python, Java) courses.
  \item Plan and facilitate weekly review sessions for the students in order to further their understanding and confidence in the language, thereby also refining my own knowledge of the fundamentals.
\end{tightemize}
\sectionsep

\runsubsection{Student Affairs at RIT}
\descript{| Orientation Supervisor}
\location{August 2018 | August 2019 | Rochester, NY}
% \vspace{\topsep} % Hacky fix for awkward extra vertical space
\begin{tightemize}
\item Responsible for the transition of around 400 new students
\item Trained 20 Orientation Leaders to lead a group of new students and helped facilitate Orientation.
\item Developed strong leadership, organization, communication, and team building skills.
\end{tightemize}
\sectionsep

% \runsubsection{RIT}
% \descript{| Coputer Science Ambassador}
% \location{September 2018 - Present | Rochester, NY}
% % \vspace{\topsep} % Hacky fix for awkward extra vertical space
% \begin{tightemize}
% \item One of the students representing the Computer Science department at RIT.
% \item Developed strong leadership and communication skills.
% \end{tightemize}
% \sectionsep

%%%%%%%%%%%%%%%%%%%%%%%%%%%%%%%%%%%%%%
%     Projects
%%%%%%%%%%%%%%%%%%%%%%%%%%%%%%%%%%%%%%

\section{Projects}
\runsubsection{Ride Board}
\location{\href{https://www.github.com/ag-ayush/rideboard}{github.com/ag-ayush/rideboard}}
A web application developed for hosting large-scale events during which users would like to carpool in order to reach their destination.
The app was built using \textbf{Flask-SQLAlchemy}, \textbf{PostgreSQL}, \textbf{Python Flask}, \textbf{Flask-pyoidc} and \textbf{bootsrap} and deployed using OpenShift.
I learned about relational databases using SQLAlchemy, database design, and hosting databases with phpPgAdmin.
\sectionsep

\runsubsection{Ride Board API}
\location{\href{https://github.com/ag-ayush/RideBoardAPI}{github.com/ag-ayush/RideBoardAPI}}
A \textbf{RESTful} HTTP API developed for the CSH Ride Board application using \textbf{Python Flask}, \textbf{Flask-SQLAlchemy}, and \textbf{Flask-pyoidc}. The responses are produced in \textbf{JSON} format. I learned about APIs and parsing JSON.
\sectionsep

\runsubsection{Smart Shades}
\location{\href{https://www.github.com/ag-ayush/smart-window-shades}{github.com/ag-ayush/smart-window-shades}}
A web service, written with \textbf{Python Flask}, \textbf{bootstrap}, \textbf{JavaScript} and \textbf{AJAX}, that allows one to control their window shades. The physical shades system was built with a Raspberry Pi and a stepper motor. I learned about JavaScript, programming using RaspberryPi and working with hardware.
\sectionsep

% \runsubsection{Big Data and Employment}
% \location{}
% Developed at ASA DataFest 2018, this was a \textbf{Java} application built to process datasets in CSV format.
% The application took the large dataset from Indeed and custom datasets found online as input to produce the states that would be the best for you to work in based on your job interest, expereince, and other factors.
% \sectionsep

% \runsubsection{WebCheckers}
% \location{}
% A \textbf{Java}-based web server for the game of checkers that allows users to play against eachother, against a computer, and spectate games. It utilizes the \textbf{Spark} web micro framework and the \textbf{FreeMarker} template engine.
% \sectionsep

% \runsubsection{Lane Follow Bot}
% \location{\href{https://www.github.com/ag-ayush/lane-follow-bot}{github.com/ag-ayush/lane-follow-bot}}
% A small software program written using \textbf{Python} and \textbf{OpenCV} to detect white lines on green grass and then output which direction to move in order to stay between the two lines. I learned how to work with basics of OpenCV.
% \sectionsep

%%%%%%%%%%%%%%%%%%%%%%%%%%%%%%%%%%%%%%
%     Volunteer
%%%%%%%%%%%%%%%%%%%%%%%%%%%%%%%%%%%%%%

% \section{Volunteer Work}
%
% \runsubsection{FIRST\textregistered \space Robotics Finger Lakes Regional}
% \location{Various Roles}
% % Helped set up and staff a 3 day FIRST\textregistered Robotics Competition regional.
% \sectionsep
%
% \runsubsection{Elementary School Tutoring}
% \location{Tutor}
% % Tutored elementary students in mathematics and science.
% \sectionsep

%%%%%%%%%%%%%%%%%%%%%%%%%%%%%%%%%%%%%%
%     AWARDS
%%%%%%%%%%%%%%%%%%%%%%%%%%%%%%%%%%%%%%

\section{Awards}
\begin{tabular}{rll}
2019         & NRS Scholar \\
2017-2019	   & Dean's List at RIT \\
2017	       & Best External Data at DataFest, RIT \\
\end{tabular}
\sectionsep

\sectionsep

\end{minipage}
\end{document}  \documentclass[]{article}
